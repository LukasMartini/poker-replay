\documentclass{article}
\usepackage[margin=1in]{geometry} % Set margins
\usepackage{hyperref} % For links
\usepackage{verbatim}

%%% "Set and Forgets"
\usepackage[utf8]{inputenc}
\usepackage[T1]{fontenc}
\usepackage{lmodern}

%%% For better newlines
\usepackage{parskip}

\begin{document}

\section*{Application Features}
    \subsection*{R6 - Content Search}

        One of the core features of our application is the ability to search for and view information about any given hand that a user has played. While conceptually simple, the implementation 
        will make use of every part of our technology stack. To find a played hand, its unique identifier must be given by the user in our web interface. 
        From there, we can query our database for results from all of the relevant tables (player\textunderscore action, player\textunderscore cards, board\textunderscore cards) 
        to return a summary of information about a hand, the session it was played in, and the actions of each player. This data is then sent back to the front-end for viewing. 

        An example of how this feature will look can be seen in our database-driven application submitted with Milestone 1.

        The query template used for every query in this feature is SELECT FROM WHERE, implemented as follows:

        \begin{verbatim}
PREPARE one_time_hand_info AS (
        SELECT poker_hand.id, poker_hand.session_id, poker_hand.total_pot, 
               poker_hand.rake, poker_hand.played_at, poker_session.table_name, 
               poker_session.game_type, poker_hand.small_blind, poker_hand.big_blind, 
               board_cards.flop_card1, board_cards.flop_card2, board_cards.flop_card3,
               board_cards.turn_card, board_cards.river_card
        FROM poker_hand, poker_session, board_cards
        WHERE poker_hand.id = $1 AND poker_hand.session_id = poker_session.id 
              AND board_cards.hand_id = $1
);
                
PREPARE player_actions_in_hand AS (
        SELECT player.id, player.name, player_action.id, player_action.hand_id, 
               player_action.action_type, player_action.amount, 
               player_action.betting_round
        FROM player, player_action
        WHERE player_action.hand_id = $1 AND player_action.player_id = player.id
);
                
PREPARE player_cards_in_hand AS (
        SELECT player.id, player.name, player_cards.hand_id, player_cards.hole_card1, 
               player_cards.hole_card2, player_cards.position, player_cards.stack_size
        FROM player, player_cards
        WHERE player_cards.hand_id = $1 AND player_cards.player_id = player.id
);
        \end{verbatim}
        with the selected data only coming from rows where the inputted hand\textunderscore id matches with the queried tables.
        Each of the three queries fulfills a different roll; the first searches for data that remains constant throughout a hand,
        the second returns a list of actions taken by each player, and the third queries for the cards each player is dealt. Additional
        contextual information is provided with each query as well to aid in preparing them for user consumption.

        If we run each of the above queries on our sample dataset using a hand\textunderscore id of 12, the results are:
        \begin{verbatim}
one_time_hand_info: 
       [[12, 2, "0.05", "0.00", "Wed, 27 Sep 2023 23:05:04 GMT", 
       "Alya III", "Cash", "0.01", "0.02", "7h", "2h", "Kc", null, null]]

player_actions_in_hand: 
       [[1, "HortonRoundtree", 153, 12, "ante", "0.01", "Preflop"],
        [2, "fourz4444", 154, 12,"ante", "0.02", "Preflop"],
        [3, "CashMatteo", 155, 12, "call", "0.02", "Preflop"],
        [6, "bouchizzle", 156, 12, "fold", null, "Preflop"],
        [4, "ljab26", 157, 12, "fold", null, "Preflop"],
        [5, "vegasricky", 158, 12, "fold", null, "Preflop"],
        [1, "HortonRoundtree", 159, 12, "fold", null, "Preflop"],
        [2, "fourz4444", 160, 12, "check", null, "Preflop"],
        [2, "fourz4444", 161, 12, "check", null, "Flop"],
        [3, "CashMatteo", 162, 12, "bet", "0.03", "Flop"],
        [2, "fourz4444", 163, 12, "fold", null, "Flop"],
        [3, "CashMatteo", 164, 12, "collect", "0.05", "Showdown"]]

player_cards_in_hand: 
       [[1, "HortonRoundtree", 12, null, null, 1, "1.71"],
        [2, "fourz4444", 12, null, null, 2, "1.62"],
        [3, "CashMatteo", 12, null, null,3, "2.41"],
        [6, "bouchizzle", 12, null, null, 4, "2.30"],
        [4, "ljab26", 12, null, null, 5, "2.60"],
        [5, "vegasricky", 12, null, null, 6, "1.71"]]

        \end{verbatim}


\end{document}